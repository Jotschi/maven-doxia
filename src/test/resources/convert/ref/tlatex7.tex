\batchmode
\newcommand{\psectioni}[1]{\section{#1}}
\newcommand{\psectionii}[1]{\subsection{#1}}
\newcommand{\psectioniii}[1]{\subsubsection{#1}}
\newcommand{\psectioniv}[1]{\paragraph{#1}}
\newcommand{\psectionv}[1]{\subparagraph{#1}}
\newcommand{\ptitle}[1]{\title{#1}}
\newcommand{\pauthor}[1]{\author{#1}}
\newcommand{\pdate}[1]{\date{#1}}
\newcommand{\pmaketitle}{\maketitle}
\newenvironment{plist}{\begin{itemize}}{\end{itemize}}
\newenvironment{pnumberedlist}{\begin{enumerate}}{\end{enumerate}}
\newcommand{\pdef}[1]{\textbf{#1}\hfill}
\newenvironment{pdefinitionlist}
{\begin{list}{}{\settowidth{\labelwidth}{\textbf{999.}}
                \setlength{\leftmargin}{\labelwidth}
                \addtolength{\leftmargin}{\labelsep}
                \renewcommand{\makelabel}{\pdef}}}
{\end{list}}
\newenvironment{pfigure}{\begin{center}}{\end{center}}
\newcommand{\pfigurecaption}[1]{\\* \vspace{\pparskipamount}
                                \textit{#1}}
\newenvironment{ptable}{\begin{center}}{\end{center}}
\newenvironment{ptablerows}[1]{\begin{tabular}{#1}}{\end{tabular}}
\newenvironment{pcell}[1]{\begin{tabular}[t]{#1}}{\end{tabular}}
\newcommand{\ptablecaption}[1]{\\* \vspace{\pparskipamount}
                               \textit{#1}}
\newenvironment{pverbatim}{\begin{small}}{\end{small}}
\newsavebox{\pbox}
\newenvironment{pverbatimbox}
{\begin{lrbox}{\pbox}\begin{minipage}{\linewidth}\begin{small}}
{\end{small}\end{minipage}\end{lrbox}\fbox{\usebox{\pbox}}}
\newcommand{\phorizontalrule}{\begin{center}
                              \rule[0.5ex]{\linewidth}{1pt}
                              \end{center}}
\newcommand{\panchor}[1]{\index{#1}}
\newcommand{\plink}[1]{#1}
\newcommand{\pitalic}[1]{\textit{#1}}
\newcommand{\pbold}[1]{\textbf{#1}}
\newcommand{\pmonospaced}[1]{\texttt{\small #1}}
\newcommand{\pfiguregraphics}[1]{\includegraphics{#1.eps}}

\documentclass[a4paper]{article}
\usepackage{a4wide}
\usepackage[english]{babel}
\usepackage{graphicx}
\usepackage{ifthen}
\usepackage{times}
\usepackage[T1]{fontenc}
\usepackage[latin1]{inputenc}

\usepackage{fancyhdr}
\pagestyle{fancy}

\newlength{\pparskipamount}
\setlength{\pparskipamount}{1ex}

\setlength{\parindent}{0pt}
\setlength{\parskip}{\pparskipamount}

\begin{document}

\ptitle{The \pbold{P}\pbold{T}\pbold{F} \pbold{F}\pitalic{ormat} and Related
\pitalic{Utilities}\panchor{Utilities}}
\pauthor{Pixware\\
Immeuble Capricorne\\
23 rue Colbert\\
78180 Montigny Le Bretonneux\\
France\\
Phone: (33) 01 30 60 07 00\\
Fax: (33) 01 30 96 05 23\\
Email: info@pixware.fr}
\pdate{9 October 1999}
\pmaketitle

\fancyhf{}
\renewcommand{\footrulewidth}{0pt}
\fancyhead[er,ol]{\pitalic{The PTF Format and Related Utilities}}
\fancyhead[el,or]{\pitalic{\thepage}}
\renewcommand{\headrulewidth}{0.4pt}

This document describes a format called \pitalic{PTF} (\pbold{P}oor
\pbold{T}ext \pbold{F}ormat\panchor{Poor Text Format}).

PTF documents are:

\begin{plist}

\item{} Readable. The PTF reader use as few markup as possible to recognize
the structure of a document. Instead this reader generally uses the
\pitalic{relative} indentation of paragraphs.

\item{} Easy to type using any text editor.

\item{} Easy to translate to real document formats such as LaTeX, troff, HTML
or RTF.

\end{plist}

PTF is a good format for writing README files and C/C++ comments. With a tool
such as docc you can automatically extract these comments from your source
files and convert the extracted PTF chunks into an acceptable reference
manual.

Send bug reports to mailto:info@pixware.fr.

\psectioni{Sections}

The title of a section must not be indented at all. The title of a section is
typically rendered using a large bold font.

Example:

\begin{pverbatim}
\begin{verbatim}
*This is a subsection

**This is a subsubsection

  First paragraph of first subsubsection.

**This is a another subsubsection

  First paragraph of the other subsubsection.

*This is another subsection

  First paragraph of the other subsection.
\end{verbatim}
\end{pverbatim}

is rendered as:

\psectionii{This is a subsection}

\psectioniii{This is a subsubsection}

First paragraph of first subsubsection.

\psectioniii{This is a another subsubsection}

First paragraph of the other subsubsection.

\psectionii{This is another subsection}

First paragraph of the other subsection.

\psectioni{Paragraphs}

Paragraphs must be indented (by an arbitrary amount). The amount of white
space between words is not important but at least one open line must be used
to specify the end of a paragraph.

Do not forget this open line after all sections, paragraphs or list items
otherwise expect very strange results.

When used to determine the indentation of a paragraph, the tab character is
counted as 8 white spaces. This means that you'd better set your favorite text
editor tab width to 8.

The maximum line width is 255 characters.

\psectioni{Lists}

\begin{plist}

\item{} List item 1. List items are indented too. Inside lists, indentation
\symbol{45}\symbol{45} \pitalic{i.e. the amount of white space found at the
beginning of the first line of a paragraph} \symbol{45}\symbol{45} must be
used consistently because it is used to find out when to end a list or when to
begin a nested sub\symbol{45}list.

\begin{pdefinitionlist}

\item[\mbox{A\pitalic{i~j}\panchor{Ai j}}] Nested labeled list item
\pbold{A}\pitalic{i~j}.

\item[\mbox{B}] Nested labeled list item \pbold{B}. Item labels are rendered
using a bold font.

[ is a special character which is used to specify labeled list items. Note
that [ is a special character only if it is the first character of an indented
paragraph.

\textbackslash may be used to quote special characters. As usual, the
character \textbackslash itself is specified by typing \textbackslash
\textbackslash .

* is another special character which is used to specify ``ticked'' list items.
The '*' tick character is generally replaced by a nicer tick symbol such as a
bullet or a little square when the document is translated into a real
typesetting system such as TeX.

Like [, * is a special character only if it is the first character of an
indented paragraph.

\end{pdefinitionlist}

\item{} List item 2.

\begin{plist}

\item{} Nested list item 1. Several '*' may be used to make list items more
readable. Only a single tick will appear in the formatted text.

\item{} Nested list item 2.

\newpage

\begin{plist}

\item{} Deeply nested.

\begin{plist}

\item{} Very deeply nested.

\end{plist}

\end{plist}

\end{plist}

\item{} List item 3.

\end{plist}

Numbered lists:

\begin{pnumberedlist}
\renewcommand{\theenumi}{\arabic{enumi}}

\item{} Decimal numbering.

\begin{pnumberedlist}
\renewcommand{\theenumii}{\Alph{enumii}}

\item{} Upper\symbol{45}alpha numbering. Item \#1.

\item{} Upper\symbol{45}alpha numbering. Item \#2.

\item{} Upper\symbol{45}alpha numbering. Item \#3.

\end{pnumberedlist}

Item \#1.

\begin{pnumberedlist}
\renewcommand{\theenumii}{\alph{enumii}}

\item{} Lower\symbol{45}alpha numbering. Item \#1.

\item{} Lower\symbol{45}alpha numbering. Item \#2.

\item{} Lower\symbol{45}alpha numbering. Item \#3.

\end{pnumberedlist}

\item{} Decimal numbering.

Item \#2.

\item{} Decimal numbering.

\begin{pnumberedlist}
\renewcommand{\theenumii}{\Roman{enumii}}

\item{} Upper\symbol{45}roman numbering. Item \#1.

\item{} Upper\symbol{45}roman numbering. Item \#2.

\item{} Upper\symbol{45}roman numbering. Item \#3.

\begin{pnumberedlist}
\renewcommand{\theenumiii}{\roman{enumiii}}

\item{} Lower\symbol{45}roman numbering. Item \#1.

\item{} Lower\symbol{45}roman numbering. Item \#2.

\item{} Lower\symbol{45}roman numbering. Item \#3.

\end{pnumberedlist}

\end{pnumberedlist}

Item \#3.

\end{pnumberedlist}

\psectioni{Fonts}

\begin{plist}

\item{} Text between \symbol{60} and \symbol{62} is rendered in an
\pitalic{italic} font.

\item{} Text between \symbol{60}\symbol{60} and \symbol{62}\symbol{62} is
rendered in a \pbold{bold} font.

\item{} Text between \symbol{60}\symbol{60}\symbol{60} and
\symbol{62}\symbol{62}\symbol{62} is rendered in a
\pmonospaced{fixed\symbol{45}pitch} font.

\end{plist}

It is not possible to nest font commands.

Do not try to specify fonts in section titles, list item labels, etc, because
this kind of text is generally associated with specific presentation
attributes.

\psectioni{Horizontal line}

A line which is not indented and which begins with at least 3 '=' (equal)
characters is rendered as an horizontal line which is as wide as a page. Past
the 3 first '=' characters, the rest of the line is ignored.

Example:

\phorizontalrule

\psectioni{Preformatted paragraphs}

It is possible to specify that a paragraph is preformatted by placing the text
between two lines which are not indented and which begin with at least 3
'\symbol{45}' (minus) characters.

This preformatted paragraph (typically source code) is rendered using a
fixed\symbol{45}pitch font.

Example:

\begin{pverbatim}
\begin{verbatim}
int main(int    argc,
         char*  argv[])
{
    puts("Hello MMMarvelous World!!!");
    return 0;
}
\end{verbatim}
\end{pverbatim}

Note that borders are drawn around preformatted paragraphs if the sequence
'\pmonospaced{+\symbol{45}\symbol{45}}' is used instead of sequence
'\pmonospaced{\symbol{45}\symbol{45}\symbol{45}}'.

Preformatted paragraph with borders may not be supported by all output
formats.

Preformatted paragraphs are always displayed at current indentation level.
Example:

\begin{plist}

\item{} List item 1

\begin{plist}

\item{} Sublist item 1

\begin{pverbatimbox}
\begin{verbatim}
 A      |       B
 -------+---------
 10     |       20
 30     |       40
\end{verbatim}
\end{pverbatimbox}

Nested figure:

\begin{pfigure}
\pfiguregraphics{pixware}
\pfigurecaption{Nested\\
figure caption.}
\end{pfigure}

Nested line:

\phorizontalrule

End of sublist.

\end{plist}

\item{} List item 2

\phorizontalrule

\end{plist}

Non nested line:

\phorizontalrule

Non nested figure:

\begin{pfigure}
\pfiguregraphics{pixware}
\end{pfigure}

No caption.

\psectioni{Tables}

This is a table without a caption:

\begin{ptable}
\begin{ptablerows}{cc}
\begin{pcell}{c}\pbold{Cell} 1\end{pcell} &
\begin{pcell}{c}\pbold{Cell 2}, \pitalic{line 1}\\
\pitalic{line 2}\\
\pitalic{line 3}\end{pcell}\\
\begin{pcell}{c}Unix\panchor{Unix} pipe = ``\textbar ''\end{pcell} &
\begin{pcell}{c}Cell 4\end{pcell}\\
\end{ptablerows}
\end{ptable}

This is another table:

\begin{ptable}
\begin{ptablerows}{|r|l|l|}
\hline
\begin{pcell}{r}\pbold{Display screen}\\
\pbold{number}\end{pcell} &
\begin{pcell}{l}\pbold{1}\end{pcell} &
\begin{pcell}{l}\pbold{2}\end{pcell}\\
\hline
\begin{pcell}{r}width\end{pcell} &
\begin{pcell}{l}1024\end{pcell} &
\begin{pcell}{l}1280\end{pcell}\\
\hline
\begin{pcell}{r}height\end{pcell} &
\begin{pcell}{l}768\end{pcell} &
\begin{pcell}{l}1024\end{pcell}\\
\hline
\begin{pcell}{r}depth\end{pcell} &
\begin{pcell}{l}16\end{pcell} &
\begin{pcell}{l}24\end{pcell}\\
\hline
\end{ptablerows}
\ptablecaption{Long, long, long, long, long, long\\
table caption.}
\end{ptable}

\psectioni{Figures}

In order to specify a figure in PTF, a file name \pitalic{without any suffix}
must be put between [ and ] \pitalic{at column 0}, optionally followed by the
caption of the figure.

If the file name is something like \pmonospaced{images/myfig}, tools such as
docc generally expect to find actual files called:

\begin{plist}

\item{} \pmonospaced{images/myfig.eps} when converting PTF to LaTeX or to
troff.

\item{} \pmonospaced{images/myfig.gif} when converting PTF to HTML.

\item{} \pmonospaced{images/myfig.bmp} when converting PTF to RTF.

\end{plist}

This command may not be supported by all output formats.

Example:

\begin{pverbatim}
\begin{verbatim}
[pixware] This is the logo of the Pixware company.
\end{verbatim}
\end{pverbatim}

is rendered as:

\begin{pfigure}
\pfiguregraphics{pixware}
\pfigurecaption{This is the logo of the Pixware company.}
\end{pfigure}

\psectioni{Hypertext links}

It is possible to specify hypertext links in PTF.

\begin{plist}

\item{} Text between \{ and \} is an hypertext link target (i.e. you can go
there if you click on an hypertext ``button'').

This text, which may be converted to an index entry by systems such as LaTeX,
should be at most a few words long.

\item{} Text between \{\{ and \}\} is an hypertext link start (i.e. a
``button'' you can click on to jump to the corresponding target).

This text must be defined somewhere else (possibly in another file
\symbol{45}\symbol{45} see the documentation of docc) as an hypertext target.

hypertext ``buttons'' are generally rendered using specific presentation
attributes so you don't need to add your own font commands.

\end{plist}

Example:

If this document is translated to HTML or to non\symbol{45}linear (winhelp)
RTF, click on Poor Text Format to jump to the introduction.

\psectioni{Comments}

Text between 2 '\textasciitilde ' (tilde) characters and the end of line is
ignored.

This is also a very convenient way to make sections stand out very cleary
above paragraphs by underlining them using lines of \textasciitilde 's.

\psectioni{More cryptic commands}

These commands are generally generated by programs such as \pmonospaced{sed}
or \pmonospaced{docx} (see the docx utility). You'll rarely need to use these
commands.

\begin{pdefinitionlist}

\item[\mbox{\textbackslash NNN}] \pmonospaced{NNN} is a 3\symbol{45}digit
octal number between 0 and 255 which represent an 8\symbol{45}bit character
belonging to the ISO Latin 1 character set.

Note that if you have a 8\symbol{45}bit text editor, you don't need to use
\textbackslash NNN commands at all because you can directly type any ISO Latin
1 character.

Example:

\begin{pverbatim}
\begin{verbatim}
This paragraph contains ISO Latin 1 characters such as 
the dollar sign \044, the copyright sign \251 or small e 
accute accent \351 which is heavily used <en fran\347ais>.
\end{verbatim}
\end{pverbatim}

is rendered as:

This paragraph contains ISO Latin 1 characters such as the dollar sign \$, the
copyright sign � or small e accute accent � which is heavily used \pitalic{en
fran�ais}.

\item[\mbox{\textbackslash \symbol{60}SPACE\symbol{62}}] (the '\textbackslash
' character followed by a space) Non breaking space.

Example:

\begin{pverbatim}
\begin{verbatim}
Word\ 1 word\ 2 word\ 3 word\ 4 word\ 5 word\ 6
word\ 7 word\ 8 word\ 9 word\ 10 word\ 11 word\ 12
word\ 13 word\ 14 word\ 15 word\ 16 word\ 17 word\ 18
word\ 19 word\ 20 word\ 21 word\ 22 word\ 23 word\ 24.
\end{verbatim}
\end{pverbatim}

is rendered as:

Word~1 word~2 word~3 word~4 word~5 word~6 word~7 word~8 word~9 word~10 word~11
word~12 word~13 word~14 word~15 word~16 word~17 word~18 word~19 word~20
word~21 word~22 word~23 word~24.

This command may not be supported by all output formats.

\item[\mbox{\textbackslash \symbol{60}CR\symbol{62}}] (the '\textbackslash '
character followed by a carriage return) Line break.

Example:

\begin{pverbatim}
\begin{verbatim}
First line.\                 
Second line.\                 
Third line.                 
\end{verbatim}
\end{pverbatim}

is rendered as:

First line.\newline
Second line.\newline
Third line.

This command may not be supported by all output formats.

\item[\mbox{[]}] List break. An indented line which contains only [] may be
used to force the end of a list. The indentation of the line specifies which
list (or nested lists) is to be ended. This command is especially useful for
preformatted paragraphs.

Example:

\begin{pverbatim}
\begin{verbatim}
                 * list 1 item 1

                 * list 1 item 2

                     ** list 2 item a

                     ** list 2 item b

                     ** list 2 item c

                 []

 +--
 Without the [], this paragraph would have been contained 
 inside list 1 item 2.
 +--
\end{verbatim}
\end{pverbatim}

is rendered as:

\begin{plist}

\item{} list 1 item 1

\item{} list 1 item 2

\begin{plist}

\item{} list 2 item a

\item{} list 2 item b

\item{} list 2 item c

\end{plist}

\end{plist}

\begin{pverbatimbox}
\begin{verbatim}
Without the [], this paragraph would have been contained 
inside list 1 item 2.
\end{verbatim}
\end{pverbatimbox}

Note that horizontal lines, figures and page breaks force all nested lists to
end.

\item[\mbox{\textasciicircum L}] (a formfeed character \pitalic{at column 0})
Page break.

This command may not be supported by all output formats.

Example:

\begin{pverbatim}
\begin{verbatim}
^L
  This paragraph should be located at the top of a new page.
\end{verbatim}
\end{pverbatim}

is rendered as:

\newpage

\end{pdefinitionlist}

This paragraph should be located at the top of a new page.

\end{document}

